% iaus2esa.tex -- sample pages for Proceedings IAU Symposium document class
% (based on v1.0 cca2esam.tex)
% v1.04 released 17 May 2004 by TechBooks
%% small changes and additions made by KAvdH/IAU 4 June 2004
% Copyright (2004) International Astronomical Union

\NeedsTeXFormat{LaTeX2e}

\documentclass{iau_FM}
\usepackage{graphicx}
\usepackage{hyperref}

\title[Andean ROAD] %% give here short title %%
{The new Andean Regional \\ Office of Astronomy for Development}

\author[Farid Char \& Jaime Forero-Romero]   %% give here short author list %%
{Farid Char$^1$
%%  \thanks{Present address: Fluid Mech Inc., 24 The Street, Lagos, Nigeria.},
 \and Jaime Forero-Romero$^2$}

\affiliation{$^1$Unidad de Astronom\'ia, Facultad de Cs. B\'asicas, Universidad de Antofagasta \\ Avenida U. de Antofagasta 02800, Antofagasta 1270300, Chile \\ tel.: +56 55 2637596, email: {\tt farid.char@uantof.cl} \\[\affilskip]
$^2$Departamento de F\'isica, Universidad de los Andes \\ Carrera 1 18-10, Bloque Ip., Bogot\'a, Colombia \\ tel.: +571 3394949-5183, email: {\tt je.forero@uniandes.edu.co}}

\pubyear{2015}
\setcounter{page}{1}
\jname{Astronomy in Focus, Volume 1} 
\editors{Piero Benvenuti, ed.}
\begin{document}

\maketitle

\begin{abstract}
The Andean Regional Office of Astronomy for Development (ROAD) is a new effort in South America to serve several goals in astronomical development. Six countries (Bolivia, Colombia, Chile, Ecuador, Per� and Venezuela) will work together, representing a common language block in the Andean region and focusing on develop strategies to strengthen the professional research, education and popularization of astronomy. Our current Working Structure comprises a ROAD Coordinator and Coordinators per Task Force, and our current plan includes several activities described in the following presentation.

\keywords{andean, office, development}
%% add here a maximum of 10 keywords, to be taken form the file <Keywords.txt>
\end{abstract}

\firstsection % if your document starts with a section,
              % remove some space above using this command.
\section{Introduction}

The countries in the Andean Region (Bolivia, Colombia, Chile, Ecuador, Peru and Venezuela) represent a common language block in South America. They share a common vision about the importance of the cultural development, that might be enhanced through a regional cooperation and institutional commitment. An Andean Regional Office of Astronomy for Development (ROAD) can serve this goal. It will strengthen ongoing collaboration efforts, create channels of communication and develop new strategies to exchange knowledge and human resources. The participating institutions of this new ROAD will explore educational activities/material to be shared between the Andean countries, standardizing the knowledge and creating inspirational experiences. We expect to generate an homogeneous activity in each Andean country, taking into account the special role of Chile in global astronomy, due to its great conditions for astronomy and the involvement of many professional observatories, universities and astronomy institutions.

\section{Overview of Task Forces}

The Andean ROAD is under supervision of the Office of Astronomy for Development (OAD) of the IAU. The main goal of the Andean ROAD is to guarantee and strengthen effective methods of communication between the representatives and coordinators of global, regional and local projects implemented in the Andean countries, through the three Task Forces established by the OAD : (i) Astronomy for Universities and Research, (ii) Astronomy for Children and Schools and (iii) Astronomy for the Public. The ROAD Main Coordinator is Jaime Forero-Romero, PhD, from the Universidad de los Andes (Bogot\'a, Colombia), and our plans includes several projects and ideas for each Task Forces, under supervision of a single representative, as follows:

\textbf{Task Force 1}. \textit{Coordinator: Germ\'an Chaparro, PhD (Universidad ECCI, Bogot\'a, Colombia)}.
- Andean School on Astronomy/Astrophysics. The main objective is to organize a school for advanced undergraduate  and graduate students. The first version of this school was held in Quito (Ecuador) during December 2014 with great success, and we expect to develop a new version during 2016, to be held in Lima (Per\'u), as a joint effort between Per\'u and Chile.

- Andean Graduate Program. The main objective is to determine the feasibility of creating and funding an Andean graduate program. This could be a great opportunity to provide powerful experiences in astronomy for those countries without formal programs.

- Massive Open Online Courses. The main objective is to create a MOOC at the introductory undergraduate level and general public. This MOOC could also have a version for the general public, and can be considered as a TF3 project, too.

\textbf{Task Force 2}. \textit{Coordinator: \'Angela P\'erez (Parque Explora, Medell\'in, Colombia)}.
- Virtual Training Sessions. The main objective is to create virtual training and activities open to teachers and students, by developing a virtual platform to hold virtual workshops on information technologies applied to astronomy and space sciences.

- Special teachings. Design teaching material to work with visually impaired students. The first step will be a comprehensive research on this kind of material and current experiences. Later on will be defined the best way to produce and distribute these materials.

- Annual TF2 meeting. An annual meeting to gather all the collaborators in the Task Force, looking for new collaborations and alliances. The venue will change every year, with a different key topic.

\textbf{Task Force 3}. \textit{Coordinator: Farid Char (Sociedad Chilena de Astronom\'ia, Chile)}.
- Development for planetariums. The main objective is to develop special shows for planetariums in the region, with the aim to highlighting ancestral traditions and the particular cosmovision from many communities in South America.

- Communicating Astronomy with the Public. The main objective is to organize periodic meeting to gather all the collaborators in the Task Force. This should be a very powerful platform for all people involved in communicating astronomy to the public. The first CAP meeting will be held in Medell\'in  (16-20 May 2016), only a few months before the XV Latin American Regional IAU Meeting, in Cartagena de Indias (3-7 October 2016).

\section{The role of Chile and future plans}

Due to the large size of the Chilean professional astronomical community and its special role in global astronomy. All the interaction to coordinate activities in Chile will be done through the Sociedad Chilena de Astronom\'ia (SOCHIAS), having designated specific representatives per Task Force, and a main coordinator of the Chilean branch. The physical address of the Chilean branch (formally Oficina Nacional de Coordinaci\'on) is located in the Unidad de Astronom\'ia of the Universidad de Antofagasta. We expect to be strongly involved in this new effort, in order to be aligned with the goals of the OAD, as established by the \cite[International Astronomical Union (2012)]{IAU2012} through its Strategic Plan 2010-2020, improving the overall state of astronomy development in our region. More information about our future plans, can be accessed through the official website of the Andean ROAD: \url{http://andean-road.uniandes.edu.co}.

\begin{thebibliography}{}

\bibitem[International Astronomical Union (2012)]{IAU2012}
{International Astronomical Union} 2012, 
\textit{Astronomy for Development Strategic Plan 2010-2020}.


\end{thebibliography}

\end{document}
